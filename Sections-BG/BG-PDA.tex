\documentclass[main.tex]{subfiles}

%%  PLANE DETECTION ALGORITHMS/ APPROACHES  %%

\begin{document}


\section{General approaches of PDA}
3 types of approaches \cite{Hulik_Spanel_Smrz_Materna_2014,Limberger_Oliveira_2015}


\subsection*{RANSAC}
\paragraph{What is Ransac, how does it work}
RANdom SAmple Consensus (RANSAC)\cite{Fischler1981RandomSC} is a very popular approach used for the parameter estimation of mathematical models in data sets with a variable number of outliers \cite{Yang_Forstner,Sun_Mordohai_2019,Oehler_Stueckler_Welle_Schulz_Behnke_2011}.
The RANSAC algorithm consists of two stages. First, a hypothesis is generated by taking a random subset of data and fitting the desired model to it. In the context of plane detection, a minimum of three points need to be sampled \cite{Yang_Forstner}. The next step is used to verify the previously calculated model. This is done by calculating both the number of points that are consistent with the model, henceforth called inliers, and the number of points that are considered outliers, i.e. points that do not lie within a certain threshold of the model. If the amount of inliers from the current model exceeds the amount of inliers of the previously best-fitting model, both the model and the amount of associated inliers are updated.
The described algorithm can be found in Algorithm~\ref{alg:RANSAC}.
\RestyleAlgo{ruled}
\SetKwComment{Comment}{/* }{ */}

\begin{algorithm}
\caption{RANSAC}\label{alg:RANSAC}
\KwData{$P \neq \emptyset$, $N > 0$}
\KwResult{\textrm{best fitting model} $m_{best}$}
$m_{best} \gets \emptyset$\;
$i_{max} \gets 0$\;
\While{$N > 0$}{
  $p \gets \textrm{random sample of}$ $P$\;
  $m \gets \textrm{fit model to}$ $p$\;
  $i \gets \textrm{amount of inliers in}$ $m$\;
  \If{$i > i_{max}$}{
    $i_{max} \gets i$\;
    $m_{best} \gets m$\;
  }
}
\end{algorithm}

\paragraph{Ransac in the context of plane detection}

\subsection*{Hough Transform}
\paragraph{What is HT, how does it work generally?}
The Hough Transform is used for the detection of parameterized objects in two- or three-dimensional space.
\paragraph{HT in the context of Plane Detection (i.e. 3D HT)}
\subsection*{Region Growing}
\textcolor{red}{I am not happy with this}
\paragraph{what is RG, how does it work}
Originally, Region Growing (RG) has been introduced as a method of image segmentation \cite{Adams_Bischof_1994}. The only necessary input to perform region growing is a set of pre-determined seeds, which can be chosen manually or by automated procedures. The general idea of the Region Growing algorithm is to gradually increase the size of regions by appending adjacent pixels depending on a membership criterion. In the original implementation, an adjacent pixel $x$ is only added to a region if all labelled neighbors of this pixel have the same label. After that, the mean gray-scale value of the entire corresponding region is updated. Lastly, all adjacent pixels of $x$ whose pixel value is also within the predetermined threshold or are still unset are added to the list.   

\paragraph{RG in the context of plane detection}


\end{document}