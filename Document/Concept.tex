\documentclass[main.tex]{subfiles}
\begin{document}


\chapter{Concept} \label{chap:Concept}

\section{Introduction}
\paragraph*{What this chapter is about}
This chapter covers...
\paragraph*{Chapter Structure}
It is Structured as follows ...

\section{Used Sensors}
\subsection*{Why these ones}
\paragraph*{open source ros wrapper realsense-ros}
\subsection*{Detailed Information i guess?}

\section{SLAM}
\subsection*{Why SLAM is needed}
\paragraph*{map building}
reduces complexity for plane detection
\subsection*{SLAM SOTA}
RTAB-MAP, ORBSLAM etc..
\paragraph*{bc we are using visual, stereo etc.}
\paragraph*{We choose RTAB-MAP bc..}
just makes sense because its already included in realsense-ros and also has a wide variety of
options regarding output formats
\section{Requirements}

\paragraph*{Problem/Use case}
We want to find planes quickly..

\subsection*{Definition Real-time}
Because camera runs at 30fps ..
Daher definieren wir in dieser arbeit "echtzeit" als

\subsection*{Precision}
What does precision mean in general \\
What is a "good" precision
Werden wir \$später definieren
fokus wird auf echtzeit gelegt wird und wir gucken was wir im rahmen der echtzeit im besten fall an präz raus holen können
dazu sind allein die angegeben werte nicht 100\%ig aussagekräftig weil größtenteils verschiedene datensätze benutzt wurden
somit werden wir in dieser arbeit einen einheitlichen vergleich von PDAs unter besonderer berücksichtigung des echtzeitkriteriums machen


\section{PLane Detection Algorithms}
\paragraph*{Intro?}
\paragraph*{SOTA}
aufgrund von \$SACHEN filtert sich die gesamte liste runter auf diese enge auswahl
\$DARUM werden wir den fokus auf diese N algorithmen legen
\section{Summary}
we want to answer the question if real time plane detection is viable on OTS hardware like the intel realsense.
to answer the q, we selected  \$SLAM which gives us \$OUTPUT which we use as input for out \$PDAs
\end{document}